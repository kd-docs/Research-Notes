\section{Compute bouncing factor from simulation result}

Bouncing factor is computed with following algorithm. This algorithm captures extra bouncing length of an agent 
trajectory scaled with reversed target distance. Bouncing is defined as any movement that is opposite to the agent 
target vector. As shown in following formula:

\begin{equation}
   \frac{1}{\mathbf{s}_{T-1}-\mathbf{s}_0} \int \mathit{abs} \left( \nabla_t\mathbf{s} \cdot \mathit{norm}\left( 
   {\mathbf{s}_{T-1}-\mathbf{s}_0} \right )\right )
\end{equation}

In above formula, \( \mathbf{s}_t \) indicates the agent position in time \( t \). \( \mathit{abs}(\cdot) \) is the 
absolute value of a scalar, and \( \mathit{norm}(\cdot) \) returns a unit vector that indicates the direction of the 
vector inside. \( \nabla_t\mathbf{s} \) is the gradient of vector \( \mathbf{s} \) over time \( t \), which is 
implemented as \( \mathbf{s}_{t+dt} - \mathbf{s}_{t} \) where \( dt = 0.05s \) is a constant value during the 
simulation.

\section{Code to compute bouncing factor}

Execution script is attached in \textattachfile{attachments/bouncing_factor.rb}{\textcolor{blue}{bouncing\_factor.rb}}
In order to replicate result, run

\begin{minted}{bash}
  ./bouncing_factor.rb --title=[TITLE] file1 .. filen
\end{minted}

It will collect all agent trajectories from trajectory binary \mintinline{bash}{file1} to \mintinline{bash}{filen} and 
show result in a histogram form.

