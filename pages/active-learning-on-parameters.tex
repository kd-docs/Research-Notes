\chapter{Extend trajectory learning work}

\section{Run simplest steersim with AI} %\begin

A clean steersuite build can be compiled inside the singularity container, 
which requires the ubuntu version of 20.04.

\begin{minted}{bash}
  $ sifpath=`path/to/steersim.sif`
  # run command with
  $ singularity run $sifpath bash
\end{minted}

This starts ubuntu 20.04 environment, every command inside 20.04 container will 
be started with \mintinline{bash}{>} instead.

\begin{minted}{bash}
  > cd build
  > premake4 gmake
  > cd gmake
  > make -j`nproc` config=release64
  # or stop at first error
  > CFLAGS=-Wfatal-errors make config=release64
\end{minted}

Then steersuite can be run as
\begin{minted}{bash}
  > cd ../bin
  > ./steersim -ai simpleAI
\end{minted}

%\end
\section{A minimal steersim config file for testing}
\todo[inline]{Fill this later}

\section{Unity's retail-ped reproduction} % \begin

Download unity project from \href{ 
https://github.com/hukaidong/RetailPeds/tree/a46f14017152f101e3e420522edc3b41b4406dbc 
}{hukaidong/RetailPeds}. Open Project, and edit scene file stored in 
'Assets/Scenes/Storer~Lab.unity'. In object \mintinline{csharp}{AgentManager} 
change \mintinline{csharp}{Base Model Type} to value 
\mintinline{csharp}{Moussaid}. Then hit play so you can get a replay of desired 
steering algorithm \cite{moussaid2011how}
\printbibliography

% \end
\section{Egocentric Steering Algorithm as Steersuite Independent module} %\begin
\todo[inline]{Fill this later}

%end
\chapter{Additional experiment for AIxVR}

\section{Run full training and benchmarking process} % \begin

\textbf{Requirement:}
\begin{enumerate}
  \item A steersuite configuration scenario
  \item A dataset for such scenario with specific distribution
    \begin{enumerate}
      \item Dataset is located in scenario spec dataset location under directory "subdir"
    \end{enumerate}
\end{enumerate}

The complete experiment will be executed (submitted by sbatch) with following command:
\begin{minted}{bash}
  ./bin/ruby_gpu_exec ./bin/af_evac_exec.rb logfile-name trial-name scene-name subdir
#eg: ./bin/ruby_gpu_exec ./bin/af_evac_exec.rb localfiles/sfrnd1 trial0 sf_1 fullrandom
\end{minted}

Following command is used in submit the command, for recording:
\begin{minted}[breaklines]{ruby}
require 'rake'

(0..9).to_a.product((2..14).to_a).each_with_index do |(t, s), i|
  sh("./bin/ruby_gpu_exec ./bin/af_evac_exec.rb localfiles/bench-report3/sfrnd#{"%02d" % i} #{t} sf_#{s} fullrandom")
end
\end{minted}


% \end
\section{Recover parameters from trained keras model} %\begin
A script will required a model configuration file and a model checkpoint. With 
a latent file, it will predict parameter in an array on {seqname, 
parameter}.\href{https://github.com/hukaidong/pkl-keras-train/tree/f97d8b71bf7a55fad0450452a32b71ddcf606fe9}{hukaidong/pkl-keras-train}

Notice that steersim requires a parameter defined in range [0, 1], but 
AgentFormer scaled parameters into [-1, 1] in \href{ 
https://github.com/hukaidong/MyAgentFormer/blob/e78cd3247d2a2691a5a8ca524b4c7b541fd2f55b/data/steersim.py#L154 
}{\mintinline{bash}{data/steersim.py:154}}.  A backward transform is performed 
at \mintinline{bash}{predict_scene.py:65}.

\begin{minted}{bash}
usage: predict_scene.py [-h] -i INPUT -C DIRERCTORY -o OUTPUT

optional arguments:
  -h, --help            show this help message and exit
  -i INPUT, --input INPUT
                        input json file
  -C DIRERCTORY, --direrctory DIRERCTORY
                        directory of the model
  -o OUTPUT, --output OUTPUT
                        output json file

example: singularity run --nv sifs/keras.sif python3 predict_scene.py -i \
            val.json -C "/dev/shm/activeloop-20230829-318223-agocho.snapshot/\
            agentformer-result/agent_former/latents" -o result.json
  
\end{minted}

% \end
\section{Get tradition trajectory measurement from config predictions} % \begin
\label{sec:resim-measure-precedure}
\begin{enumerate}
  \item Step 1. Generate a new set of trajectory $T_0$, store their 
    benchmarking data $B_0$.
  \item Step 2. Get latent data from trajectory using trained AgentFormer
  \item Step 3. Make prediction from latent data using trained CPN $P_1$ or 
    Plain network $P_2$
  \item Step 4. Simulate predictions, store simulated benchmarking data $B_{1, 
    2}$. Each $B_0$ will have exact one matched $B_1$ or $B_2$
  \item Step 5. For a specific metric $m$, compute MAE from $B_0, B_1$ pairs or 
    $B_0, B_2$ pairs
  \item Report result
\end{enumerate}

% \end
\section{Simulate from predictions \& Comparing database} % \begin 

This is the executable script prepared for precedure described in \fref{sec:resim-measure-precedure}. A new script was 
written in order to simulate all predicted trajectories and holds their relation at the same time, run command in 
ActiveLoop project. In current commit \mintinline{bash}{`08f785a8fd`}, this should be automatically called at the end of 
script \mintinline{bash}{./bin/af_evac_exec.rb}

\begin{minted}{ruby}
# [param]trail_name: string, usually a number indicates the repetation number of experiment
# [param]scene_name: string, trail part of scene specification after "scene_evac_" (eg sf_1)
# [param]subdir: dataset subdir used to seperate different pararmeter distributions in the
#           same configuration space.  eg: (identity, fullrandom) depends on dataset setting
# [param]snapshot_path: snapshot path for data in learning phase. used to recover model
#           in inference phase
def predict_and_benchmarks(trail_name, scene_name, subdir,
                           snapshot_path=Snapshot::SNAPSHOT_PATH)
  command = +"rvm 3.0 do ruby -Ilib ./bin/prediction_resimulate.rb "
  command << "#{trial_name},#{scene_name},#{subdir},#{snapshot_path}"
  sh(command)
end
\end{minted}

% \end -section
